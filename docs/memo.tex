% Created 2021-09-27 Mon 23:04
% Intended LaTeX compiler: pdflatex
\documentclass[11pt]{article}
\usepackage[utf8]{inputenc}
\usepackage[T1]{fontenc}
\usepackage{graphicx}
\usepackage{grffile}
\usepackage{longtable}
\usepackage{wrapfig}
\usepackage{rotating}
\usepackage[normalem]{ulem}
\usepackage{amsmath}
\usepackage{textcomp}
\usepackage{amssymb}
\usepackage{capt-of}
\usepackage{hyperref}
\author{Jin SANO}
\date{\today}
\title{Design and implementation of a pure functional language with GDT (Graph Data Type)}
\hypersetup{
 pdfauthor={Jin SANO},
 pdftitle={Design and implementation of a pure functional language with GDT (Graph Data Type)},
 pdfkeywords={},
 pdfsubject={},
 pdfcreator={Emacs 27.2 (Org mode 9.4.4)}, 
 pdflang={English}}
\begin{document}

\maketitle
\tableofcontents


\section{Introduction}
\label{sec:orga5b7697}

Functional languages feature only tree (ADT) as a data.
\begin{itemize}
\item \texttt{let rec} can create a cyclic data, but it is limited and cannot be pattern-matched as a graph.
\begin{itemize}
\item i.e. it (just) creates an infinite graph in a user point of view
\item (<-> implementation. Especially, the virtual machine).
\end{itemize}
\end{itemize}


However, we often desire of using a more powerful data structures (e.g. Queue)
for the efficiency in the execution of a program
or to write program easily.
Thus, we introduced GDT, graph data type, which is an extension of the former ADT (algebraic data type).

There are two main difficulties on achieving this.
\begin{enumerate}
\item How should we \textbf{implement a pure with graph} as a primary data structure (or can't we?).
\item How can we apply the \textbf{type system} (or can't we?)
\end{enumerate}


\subsection{The design choice we made in GDT (graph data type)}
\label{sec:orge66adc9}

There are many choices in designing this.
\begin{enumerate}
\item Is Directed or Not-directed?
\item Is rooted?
\item Is DAG only?
\item Is hyperlink allowed?
\item Is (the incoming) link ordered?
\begin{itemize}
\item For example, should we distinguish the graph which is
a. Node \texttt{A} and \texttt{B} are pointing \texttt{C}. i.e. \texttt{(A(x), B(x), x -> c)}.
b. Node \texttt{B} and \texttt{A} are pointing \texttt{C}. i.e. \texttt{(B(x), A(x), x -> C)}.
\end{itemize}
\end{enumerate}


I chose the following graph AT THIS TIME
(I possibly change my mind later. The final goal may be the (hyper)lmntal graph).
\begin{enumerate}
\item Directed (since ADT (=tree) is directed)
\item Rooted.
\item Only DAG is allowed in the pattern matching.
\begin{itemize}
\item This is JUST A DESIGN CHOICE.
\begin{itemize}
\item We will possibly extend it later.
\end{itemize}
\item Note cycle is still allowed in the generated graph (i.e. GDT is a super set of ADT)
\end{itemize}
\item Hypergraph
(if we disallow hyperlinks, we can only treat tree,
since the incoming link is allowed to write once for each node due to the constraint below)
\item Incoming link is not ordered.
\begin{itemize}
\item It is allowed to write once for each node. e.g. \texttt{x -> P(...)}.
\item c.f. ADT (= tree) has only one (unordered) incoming link.
\end{itemize}
\end{enumerate}


\subsection{Possible usecases of GDT}
\label{sec:orgb686864}
Our primary motivative example is a Queue (FOR NOW).
\begin{itemize}
\item Queue should be able to add (and possibly access and remove) an element to the last of it with cost \(O(1)\).
\end{itemize}


There are two ways to implement this in the former functional languages.
\begin{enumerate}
\item We can implement this using two stacks (lists).
\begin{itemize}
\item It anables to add/access/remove the last element in \(O(1)\), 
if we amortize the costs.
\item However, it cannot always sure the cost (\(O(1)\)).
\item and it generates intermediate data
(we need two stacks. in other words, we should pay DOUBLE cost).
\end{itemize}
\item We can implmenent this in IMPERATIVE manner (with side-effects).
\begin{itemize}
\item This is not pure and is not welcomed.
\begin{itemize}
\item side-effects makes things difficult for a programmer.
\item side-effects makes it difficult to read a program.
\item side-effects makes it difficult to test (Unit test, Integration test).
\end{itemize}
\end{itemize}
\end{enumerate}


GDT let us to define and use a \(O(1)\) and pure queue, which was not possible
in the former languages (not just the functinal languages but ALL).

The other usecase is a skip-list.
(I am looking for other examples. Please help me)
\end{document}
